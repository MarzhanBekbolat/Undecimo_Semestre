
\documentclass[journal]{IEEEtran}

\usepackage{graphicx}
\usepackage{amssymb,amsmath}
\usepackage{url}

% correct bad hyphenation here
\hyphenation{}

\usepackage[utf8]{inputenc}	

\ifCLASSOPTIONcompsoc
\usepackage[caption=false,font=normalsize,labelfon
t=sf,textfont=sf]{subfig}
\else
\usepackage[caption=false,font=footnotesize]{subfi
g}
\fi


\begin{document}
\title{}
\author{}

% The paper headers
\markboth{}%
{Shell \MakeLowercase{\textit{et al.}}: Bare Demo of IEEEtran.cls for Journals}

% make the title area
\maketitle

\begin{abstract}

\end{abstract}

\begin{IEEEkeywords}

\end{IEEEkeywords}

\IEEEpeerreviewmaketitle

\section{Introducción}


%\begin{figure*}[H]%[!t]
%\centering
%	\includegraphics[width=0.3\textwidth]{./f2d.png}
%	\caption{Circuitos usados para la caracterización de los transistores JFET (izq.) y MOSFET (der.). }
%	
%	\label{conf_jfet}
%
%\end{figure*}




%\begin{table}[!t]
%	%% increase table row spacing, adjust to taste
%	\renewcommand{\arraystretch}{1.1}
%	% if using array.sty, it might be a good idea to tweak the value of
%	 %\extrarowheight as needed to properly center the text within the cells
%	\caption{Parámetros de tensión medidos en ambos negadores.}
%	\label{t1}
%	\centering
%	%% Some packages, such as MDW tools, offer better commands for making tables
%	%% than the plain LaTeX2e tabular which is used here.
%	\begin{tabular}{|c||c||c|}
%	\hline
%	\bf Parámetro & \bf 74LS04 & \bf CMOS \\
%	\hline
%	
%	\end{tabular}
%\end{table}


% Can use something like this to pCaut references on a page
% by themselves when using endfloat and the captionsoff option.
%\ifCLASSOPTIONcaptionsoff
%  \newpage
%\fi


%%%%%%%%%%%%%% BIBLIOGRAF\'IA %%%%%%%%%%%%%%%%%%%
\begin{thebibliography}{1}

\bibitem{cd4007}
Fairchild Semiconductor. CD4007C -- Dual complementary pair plus inverter. \url{http://cva.stanford.edu/classes/cs99s/datasheets/CD4007C.pdf}



\end{thebibliography}

%\newpage
%%%%%%%%%%%%%%%%%% FIGURAS %%%%%%%%%%%%%%%%%%%%%


%\begin{figure*}[h]%[!t]
%\centering
%	\includegraphics[width=13cm]{./fig6a.png}
%	\caption{Controlador proporcional integral.}
%	
%	\label{parte2a}
%
%\end{figure*}


\end{document}

