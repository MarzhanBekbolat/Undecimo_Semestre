\documentclass[11pt,graphicx,caption,rotating]{article}
\textheight=24cm
\textwidth=18cm
\topmargin=-2cm
\oddsidemargin=0cm
%\usepackage[utf8x]{inputenc}
\usepackage[latin1]{inputenc}
\usepackage[activeacute,spanish]{babel}
\usepackage{amssymb,amsfonts}
\usepackage[tbtags]{amsmath}
\usepackage{pict2e}
\usepackage{ucs}
\usepackage{float}
\usepackage[all]{xy}
\usepackage{graphics,graphicx,color,colortbl}
\usepackage{times}
\usepackage{subfigure}
\usepackage{wrapfig}
\usepackage{multicol}
\usepackage{cite}
\usepackage{url}
\usepackage[tbtags]{amsmath}
\usepackage{amsmath,amssymb,amsfonts,amsbsy}
\usepackage{bm}
\usepackage{algorithm}
\usepackage{algorithmic}
\usepackage[centerlast, small]{caption}
\usepackage[colorlinks=true, citecolor=black, linkcolor=black, urlcolor=black,breaklinks=true]{hyperref}
\hyphenation{ele-men-tos he-rra-mi-en-ta cons-tru-yen trans-fe-ren-ci-a pro-pu-es-tas si-mu-lar vi-sua-li-za-cion}

\begin{document}
\title{\textbf{Diodos para hacer osciladores}}
\author{David Ricardo Mart�nez Hern�ndez \textbf{C�digo}:$261931$}
\date{}
\maketitle

\section{Diodo GUNN}
\noindent
Es una forma de diodo usado en la electr�nica de alta frecuencia. El principio de funcionamiento del diodo es el denominado \textbf{efecto gunn}; para que se d� este efecto, el material semiconductor debe tener una propiedad importante: \textbf{que tenga dos bandas de energ�a muy cercanas en la banda de conducci�n}. A diferencia de los diodos ordinarios construidos con regiones tipo P o N, solamente tiene regiones del tipo N.\\
Posee tres regiones:
\begin{itemize}
 \item Dos de ellas tienen regiones N fuertemente dopadas.
 \item Una regi�n delgada intermedia de material ligeramente dopado.
\end{itemize}
\noindent
Los diodos Gunn son usados para construir osciladores en el rango de frecuencias comprendido entre los $10\, GHz$ y frecuencias a�n m�s altas (hasta $THz$). Este diodo se usa en combinaci�n con circuitos resonantes construidos con gu�as de ondas, cavidades coaxiales y resonadores.\\
Los diodos Gunn suelen fabricarse de arseniuro de galio para osciladores de hasta $200\, GHz$, mientras que los de Nitruro de Galio pueden alcanzar los $3\, THz$.

\section{Diodo Impatt}
\noindent
Tambi�n conocido como diodo ``read'', pertenece tambi�n a los llamados semiconductores osciladores de ``resistencia negativa''; funciona estando conectado de forma inversa y cerca del voltaje de ruptura, entonces se produce en �l una avalancha de electrones aumentando tanto la corriente como el voltaje, hasta llegar a un punto en el cual se presenta una ``resistencia negativa'' y en conjunto con el circuito de resonancia producen oscilaciones a altas frecuencias ($3-100\, GHz$). Este se utiliza para transmisiones y radares.\\
Si un electr�n libre con suficiente energ�a golpea un �tomo de silicio, se puede romper la uni�n covalente de silicio y liberar un electr�n del enlace covalente. Si las ganancias de la energ�a de electrones liberados por estar en un campo el�ctrico y libera otros electrones de otros enlaces covalentes a continuaci�n, este proceso puede conectar en cascada muy r�pidamente en una reacci�n en cadena de la producci�n de un gran n�mero de electrones y un gran flujo de corriente. Este fen�meno se llama efecto avalancha.



\bibliographystyle{ieeetran}
\begin{thebibliography}{99}

\bibitem{page1} Dispositivos Activos de Microondas II: Osciladores. Sitio web ``\url{http://dspace.unav.es/dspace/bitstream/10171/18746/7/Tema5_DispositivosActivosII_2009v1.pdf}'', visitada el 20 de Enero de 2014.

\bibitem{page2} Gunn Diodoe. Sitio web ``\url{http://en.wikipedia.org/wiki/Gunn_diode}'', visitada el 20 de Enero de 2014.

\bibitem{page3} Impatt Diode. Sitio web ``\url{http://en.wikipedia.org/wiki/Impatt_diode}'', visitada el 20 de Enero de 2014.
\end{thebibliography}
\end{document}