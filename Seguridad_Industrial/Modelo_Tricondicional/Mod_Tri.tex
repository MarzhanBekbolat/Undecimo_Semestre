\documentclass[11pt,graphicx,caption,rotating]{article}
\textheight=24cm
\textwidth=18cm
\topmargin=-2cm
\oddsidemargin=-.5cm
\usepackage[utf8x]{inputenc}
\usepackage[activeacute,spanish]{babel}
\usepackage{amssymb,amsfonts}
\usepackage[tbtags]{amsmath}
\usepackage{pict2e}
\usepackage{float}
\usepackage[all]{xy}
\usepackage{graphics,graphicx,color,colortbl}
\usepackage{times}
\usepackage{subfigure}
\usepackage{wrapfig}
\usepackage{multicol}
\usepackage{cite}
\usepackage{url}
\usepackage[tbtags]{amsmath}
\usepackage{amsmath,amssymb,amsfonts,amsbsy}
\usepackage{bm}
\usepackage{algorithm}
\usepackage{algorithmic}
\usepackage[centerlast, small]{caption}
\usepackage[colorlinks=true, citecolor=blue, linkcolor=blue, urlcolor=blue, breaklinks=true]{hyperref}
\hyphenation{ele-men-tos he-rra-mi-en-ta cons-tru-yen trans-fe-ren-ci-a pro-pu-es-tas si-mu-lar vi-sua-li-za-cion}

\begin{document}
\title{{\huge \bf Modelo Tricondicional}}
\author{Luisa Fernanda Carvajal Ramírez \textbf{Código:} $245103$\\
	Juan David Franco Cañón \textbf{Código:} $223454$\\
	David Ricardo Martínez Hernández \textbf{Código:} $261931$\\
	Sergio Pérez Hernández \textbf{Código:} $245046$\\
	Edwin Fernando Pineda Vargas \textbf{Código:} $262100$\\
	\href{}{fcarvajalr@unal.edu.co}, \href{}{jdfrancoc@unal.edu.co}, \href{}{drmartinezhe@unal.edu.co}, \href{}{sperezh@unal.edu.co},\\ \href{}{efpinedava@unal.edu.co}}	
\date{}
\floatname{algorithm}{Algoritmo}
\maketitle
\noindent
De acuerdo con la Teoría Tricondicional del Comportamiento Seguro, para que una persona trabaje seguro deben darse tres condiciones:
\begin{enumerate}
 \item Debe poder trabajar seguro.
 \item Debe saber trabajar seguro.
 \item Debe querer trabajar seguro.
\end{enumerate}
\noindent
Para el poder trabajar seguro es necesario brindar los mejores espacios que se puedan dar en el ámbito industrial colombiano. Para que la gente pueda trabajar con seguridad las máquinas han de ser seguras, y los espacios de trabajo, los materiales y los ambientes razonablemente seguros y saludables.\\\\
La segunda condición todos los miembros de una empresa necesitan saber cómo hacer el trabajo seguro y cómo afrontar los riesgos remanentes en su contexto de trabajo. Por ello todos los empleados se informaran y formaran en seguridad laboral. Se identificaran los riesgos propios del sector, contexto, tecnología y métodos de trabajo utilizados y detectar las señales o indicios de riesgos anómalos o inminentes en el contexto de trabajo. Se abordaran los riesgos para evitar sus efectos y minimizar tanto su probabilidad de materialización como sus posibles daños. Esto se realizara dando unas capacitaciones o sugiriendo a la empresa que las realice una vez al mes.\\\\
La tercera condición del modelo depende de la motivación o los motivos para hacerlo. Esto se realizara por medio de motivaciones emocionales dando estímulos o premios, estos premios se coordinaran con la empresa dependiendo de los recursos que se dispongan en dicho momento 

\begin{thebibliography}{99}

 \bibitem{melia} Meliá, José L.
 {\em "`Seguridad Basada en el Comportamiento"'}.
 Unitat d’Investigació de Psicometria, Universidad de Valencia.

\end{thebibliography}
\end{document}