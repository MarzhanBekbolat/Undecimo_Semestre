\documentclass[12pt,graphicx,caption,rotating]{article}
\textheight=24cm
\textwidth=18cm
\topmargin=-2cm
\oddsidemargin=0cm
\usepackage[utf8x]{inputenc}
\usepackage[activeacute,spanish]{babel}
\usepackage{amssymb,amsfonts}
\usepackage[tbtags]{amsmath}
\usepackage{pict2e}
\usepackage{float}
\usepackage[all]{xy}
\usepackage{graphics,graphicx,color,colortbl}
\usepackage{times}
\usepackage{subfigure}
\usepackage{wrapfig}
\usepackage{multicol}
\usepackage{cite}
\usepackage{url}
\usepackage[tbtags]{amsmath}
\usepackage{amsmath,amssymb,amsfonts,amsbsy}
\usepackage{bm}
\usepackage{algorithm}
\usepackage{algorithmic}
\usepackage[centerlast, small]{caption}
\usepackage[colorlinks=true, citecolor=blue, linkcolor=red, urlcolor=blue, breaklinks=true]{hyperref}
\hyphenation{ele-men-tos he-rra-mi-en-ta cons-tru-yen trans-fe-ren-ci-a pro-pu-es-tas si-mu-lar vi-sua-li-za-cion}

\title{Marco Teórico}
\author{David Martínez}
\date{}
\begin{document}

\maketitle

\tableofcontents
\newpage
\section{Marco Teórico}
\subsection{Máquina Dobladora}
\noindent
Las Máquina plegadora hidráulica también conocida como prensa o Prensa de plegado, es una herramienta o máquina para doblar hojas y placas, comúnmente de material metálico. La Máquina plegadora hidráulica se pueden dividir en
\begin{itemize}
 \item Máquinas plegadoras manuales.
 \item Máquinas plegadoras hidráulicas.
 \item Máquinas plegadoras neumáticas.
 \item máquina plegadoras CNC.
\end{itemize}
\noindent
De conformidad con la mecanización de diferentes materiales, se pueden dividir en
\begin{itemize}
 \item Máquina plegadora hidráulica de metal.
 \item Máquina plegadora hidráulica de acero inoxidable.
 \item Máquina plegadora hidráulica de aluminio.
\end{itemize}
\noindent
La Máquina plegadora hidráulica se puede dividir de acuerdo a la sincronización en
\begin{itemize}
 \item Sincronización del eje de torsión.
 \item Sincronización mecánica e hidráulica.
 \begin{itemize}
  \item Movimiento hacia arriba.
  \item Movimiento hacia abajo.
 \end{itemize}
 \item Sincronización electro-hidráulica.
\end{itemize}
\noindent
La Máquina plegadora hidráulica incluye el marco, plataforma y placas de sujeción. En cuanto a la base y las barras de sujeción, la base es conformada por las conexiones de la bisagra y la placa de sujeción, integrado por la bobina y la cubierta, la bobina colocada dentro de la carcasa de asiento cóncavo, en la depresión de la parte superior de la tapa de cubierta. Utilizando el cable de la bobina de alimentación, la energía generada desde la platina por gravedad, de esta manera se logra una placa delgada entre la placa base y de sujeción.\\
Como resultado de la fuerza electromagnética en la abrazadera, la placa se puede hacer en una variedad de piezas necesarias, y pueden tener la pared lateral de la pieza era fácil de manejar.\\
Los moldes de la Máquina plegadora hidráulica automática se pueden cambiar para satisfacer las distintas necesidades \cite{page2}.

\subsection{Cortadora Hidráulica}
\noindent
La máquina tradicional venía al menos con tres herramientas principales
\begin{itemize}
 \item un punzón para punzonar agujeros en placa o en ángulo de hierro.
 \item una cizalla de barra plana para cizallar placa usualmente hasta de $24\, pulgadas\, \left( 61\, cm\right)$ de ancho.
 \item una cizalla en ángulo para cizallar ángulo de hierro a lo largo.\cite{page3}
\end{itemize}
\noindent
La máquina cortadora hidráulica es construida sobre un marco estable soldado específicamente para fiabilidad en resistencia y rigidez. La máquina es equipada con un dispositivo medidor de motor de tracción completo con calibración manual. Un dispositivo a bordo liviano ayuda a las operaciones de corte mientras que una barrera de alta sensibilidad proporciona al operario tranquilidad. La máquina también es equipada con un mecanismo conveniente para rápidos ajustes de la hoja de liquidación sobre la marcha \cite{page4}.

\subsection{Soldadura Autógena}
\noindent
La soldadura por combustión (autógena) es un procedimiento de soldadura homogénea. Esta soldadura se realiza llevando hasta la temperatura de fusión de los bordes de la pieza a unir mediante el calor que produce la llama oxiacetilénica que se produce en la combustión de un gas combustible mezclándolo con gas carburante (temperatura próxima a $3055\,°C$).\\
Se trata de un proceso de soldadura con fusión, normalmente sin aporte externo de material metálico. Es posible soldar casi cualquier metal de uso industrial
\begin{itemize}
 \item Cobre y sus aleaciones.
 \item Magnesio y sus aleaciones.
 \item Aluminio y sus aleaciones.
 \item Aceros al carbono.
 \item Aleados e inoxidables.
\end{itemize}
\noindent
Aunque actualmente ha sido desplazada casi por completo por la soldadura por arco, ya que uno de los problemas que plantea la soldadura oxiacetilénica son las impurezas que introduce en el baño de fusión además de baja productividad y difícil automatización.\cite{page7}

\subsubsection{Acetileno}
\noindent
Acetileno es el compuesto químico con la fórmula $C_{2}H_{2}$. Es un hidrocarburo y el alquino más simple. Este gas incoloro se utiliza ampliamente como un combustible y un bloque de construcción química. Es inestable en forma pura y por lo tanto por lo general se aplica como una solución. El acetileno puro es inodoro, pero los grados comerciales por lo general tienen un olor marcado debido a las impurezas.\\
Como un alquino, acetileno es insaturado porque sus dos átomos de carbono están unidos entre sí en un triple enlace. El triple enlace carbono-carbono coloca los cuatro átomos en la misma línea recta, con ángulos de enlace de $180°$.\cite{page6}

\subsubsection{Oxígeno}
\noindent
Es un miembro del grupo de calcógeno en la tabla periódica y es un elemento no metálico altamente reactivo y agente oxidante que se forma fácilmente compuestos (en particular óxidos) con la mayoría de los elementos. Por masa, el oxígeno es el elemento tercero-más abundante en el universo, después del hidrógeno y el helio, dos átomos del elemento se unen para formar el dioxígeno, un gas diatómico que es incoloro, inodoro e insípido, con la fórmula $O_2$.\cite{page7}

\subsection{Guantes}
\noindent
Es una prenda, cuya finalidad es la de proteger las manos o el producto que se vaya a manipular.\cite{ntc}

\subsection{Cascos}
\noindent
Elemento protector de la cabeza humana, o parte de ella, contraimpactos, partículas volantes, riesgos eléctricos, salpicaduras de sustancias químicas peligrosas,sustancias ígneas, calor radiante y efectos de las llamas. Se compone de un casquete, un ala ouna visera y un arnés.\cite{ntc1523}\\
Los cascos se han hecho a partir de una amplia gama de materiales , incluyendo diversos metales , plásticos, cuero , e incluso algunos materiales fibrosos tales como el Kevlar.

\subsection{Protector auditivo}
\noindent
Dispositivo que se utiliza para evitar los efectos perjudiciales del sonido en el sistema auditivo.\cite{ntc2272}

\subsubsection{Tapón auditivo}
\noindent
Protector auditivo que se coloca dentro del canal del oído externo (auditivo), o en la concha del oído, para impedir la entrada al canal del oído externo (semiauditivo).\cite{ntc2272}

\subsubsection{Orejera}
\noindent
Protector auditivo compuesto por lo general por una banda para la cabeza y dos recubrimientos (earcups) con un anillo exterior suave, cuyo fin es permitir un ajuste cómodo contra el pabellón de la oreja (supra-auditivo) o los lados de la cabeza alrededor del pabellón de la oreja (circunauditivo).\cite{ntc2272}

\subsection{Botas}
\noindent
Es un tipo específico de calzado. La mayoría de las botas cubren principalmente el pie y el tobillo y se extienden hasta la pierna, a veces hasta la rodilla o incluso la cadera. Originalmente fueron diseñadas como calzado de trabajo. La mayoría de las botas tienen un talón que se distingue claramente del resto de la suela, incluso si los dos están hechas de una sola pieza. Tradicionalmente hecha de cuero o de goma, existen infinidad de variedades según altura, color, material y estilo.

\subsubsection{Puntera protectora}
\noindent
Elemento que por su forma y características de resistencia al impacto y
y a la compresión protege los dedos del pie.\cite{ntc2257}

\subsubsection{Entresuela}
\noindent
Elemento de refuerzo adaptado en su forma a la suela del zapato, de material
resistente al esfuerzo de punción, que incorporado al calzado de seguridad protege la planta del
pie de heridas debidas a objetos puntiagudos.\cite{ntc2257}

\subsubsection{Lote}
\noindent
Cantidad de punteras o entresuelas de características similares fabricadas en
condiciones presumiblemente uniformes que se someten a inspección como un conjunto unitario.\cite{ntc2257}

\subsubsection{Muestra}
\noindent
Grupo de punteras o entresuelas extraído de un lote, que sirven para obtener la
información que permita apreciar una o más características para decidir sobre el mismo o el
proceso que lo produjo.\cite{ntc2257}



\subsection{Overol}
\noindent




\begin{thebibliography}{99}
 %Dobladoras
 \bibitem{page1} \url{http://www.fablamp.com/mu250p-sp.pdf}
 \bibitem{page2} \url{http://huatelimachine.es/1-press-brake.html}
 %Crtadoras
 \bibitem{page3} \url{http://www.thefabricator.com/article/Array/la-maquina-cortadora\\-multiusos-ironworker-esta-lista-para-la--produccian}
 \bibitem{page4} \url{http://foam-spraying.es/7-6-1-hydraulic-shearing-machine.html}
 %Soldadura
 \bibitem{page5} \url{http://es.wikipedia.org/wiki/Soldadura_por_combusti\%C3\%B3n\_\%28aut\%C3\%B3gena\%29}
 \bibitem{page6} \url{http://en.wikipedia.org/wiki/Acetylene}
 \bibitem{page7} \url{http://en.wikipedia.org/wiki/Oxygen}
 %Guantes
 %\bibitem{page8} \url{http://www.arlsura.com/index.php/component/content/article/75-centro-de-documentacion-anterior/equipos-de-proteccion-individual-/397--sp-26394}
 %\bibitem{page9} \url{http://www.confeccionestirma.com/seguridad.htm}
 %\bibitem{page10} \url{http://en.wikipedia.org/wiki/Glove}
 \bibitem{ntc1836} NTC 1836
 \bibitem{ntc2219} NTC 2219
 %Cascos
 %\bibitem{page11} \url{http://en.wikipedia.org/wiki/Helmet}
 \bibitem{ntc1523} NTC 1523
 %Tapa oidos
 %\bibitem{page12} \url{http://en.wikipedia.org/wiki/Earplug}
 \bibitem{ntc2272} NTC 2272
 %Botas
 %\bibitem{page13} \url{http://en.wikipedia.org/wiki/Boot}
  \bibitem{ntc2257} NTC 2257
 \bibitem{ntc1741} NTC 1741
 %Overol y cuerpo
 \bibitem{ntc2021} NTC 2021
 \bibitem{ntc2037} NTC 2037N
 %2.PROTECCION FACIAL  U OCULAR:
 \bibitem{ntc1771} NTC 1771
 \bibitem{ntc1825} NTC 1825
 \bibitem{ntc1826} NTC 1826
 \bibitem{ntc1827} NTC 1827
 \bibitem{ntc1834} NTC 1834
 \bibitem{ntc1835} NTC 1835
 \bibitem{ntc1836} NTC 1836
 %PROTECCION RESPIRATORIA: 
 \bibitem{ntc1584} NTC 1584
 \bibitem{ntc1728} NTC 1728
 \bibitem{ntc1589} NTC 1589
 \bibitem{ntc1733} NTC 1733
 
\end{thebibliography}
\end{document}