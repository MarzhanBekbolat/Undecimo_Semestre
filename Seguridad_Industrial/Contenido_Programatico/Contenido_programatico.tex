\documentclass[11pt,graphicx,caption,rotating]{article}
\textheight=24cm
\textwidth=18cm
\topmargin=-2cm
\oddsidemargin=0cm
\usepackage[utf8x]{inputenc}
\usepackage[activeacute,spanish]{babel}
\usepackage{amssymb,amsfonts}
\usepackage[tbtags]{amsmath}
\usepackage{pict2e}
\usepackage{float}
\usepackage[all]{xy}
\usepackage{graphics,graphicx,color,colortbl}
\usepackage{times}
\usepackage{subfigure}
\usepackage{wrapfig}
\usepackage{multicol}
\usepackage{cite}
\usepackage{url}
\usepackage[tbtags]{amsmath}
\usepackage{amsmath,amssymb,amsfonts,amsbsy}
\usepackage{bm}
\usepackage{algorithm}
\usepackage{algorithmic}
\usepackage[centerlast, small]{caption}
\usepackage[colorlinks=true, citecolor=black, linkcolor=black, urlcolor=black,breaklinks=true]{hyperref}
\hyphenation{ele-men-tos he-rra-mi-en-ta cons-tru-yen trans-fe-ren-ci-a pro-pu-es-tas si-mu-lar vi-sua-li-za-cion}

\begin{document}
\title{{\huge Contenido Programático}}
\author{Luisa Fernanda Carvajal Ramírez \textbf{Código:} $245103$\\
	Juan David Franco Cañón \textbf{Código:} $223454$\\
	David Ricardo Martínez Hernández \textbf{Código:} $261931$\\
	Sergio Pérez Hernández \textbf{Código:} $245046$\\
	Edwin Fernando Pineda Vargas \textbf{Código:} $262100$\\
	\href{}{fcarvajalr@unal.edu.co}, \href{}{jdfrancoc@unal.edu.co}, \href{}{drmartinezhe@unal.edu.co}, \href{}{sperezh@unal.edu.co},\\ \href{}{efpinedava@unal.edu.co}}	
\date{}
\floatname{algorithm}{Algoritmo}
\maketitle

\begin{abstract}
\noindent
A continuación se presenta el plan de trabajo para efectuar el análisis de riesgo en una empresa metalmecánica \textbf{``Peralta Perfilería S.A.S''}. Los contenidos, las actividades y los elementos a evaluar, irán aumentando su número, complejidad y detalle, en la medida en que avanza la asignatura de Seguridad Industrial y el proceso de análisis de riesgo de la empresa en mención.
\end{abstract}

\section{Introducción}
\noindent
La Seguridad Industrial son un conjunto de principios, criterios y normas formuladas con el objetivo es controlar el riesgo de accidentes y daños a equipos y personas en el desarrollo de una actividad productiva.\\
Su principio se atribuye a la época de la revolución industrial donde  se vio la necesidad de implementar parámetros de regulación en el desarrollo por la alta mortalidad en las fábricas. La primera ley relacionada con la seguridad industrial se hizo en Alemania en $1885$ aplicando solo para enfermedades, luego en $1911$ se aplica en Wisconsin la primera  ley de indemnización por accidentes y enfermedades sin importar si el trabajador tuvo la culpa o no, continuamente se han realizado mayores esfuerzos para aumentar la seguridad en la industria.\\
En el presente informe se darán las directrices primordiales del estudio de la seguridad industrial en la empresa \textbf{Peralta Perfileria S.A.S} del sector metalmecánico, se definirá los objetivos el alcance y cronograma para la realización del proyecto.\\
El sector metalmecánico es de gran importancia para la industria mundial por la cantidad de productos que ofrece a bajo costo, las diversas y encantadoras formas que tienen los mismos y la versatilidad en su proceso de fabricación y servicios que prestan. Para elaborar sus productos, la metalmecánica posee $4$ etapas:
\begin{enumerate}
  \item Corte.
  \item Doblaje.
  \item Troquelado.
  \item Armado.
\end{enumerate}
\noindent
Los riesgos en estas etapas pueden ocasionar cortaduras leves, golpes simples o quemaduras superficiales, hasta desmembramientos, quemaduras graves, serias lesiones o la muerte. Por ende, es urgente realizar un análisis de riesgo detallado y riguroso en seguridad industrial para este campo.

\section{Objetivos}
\noindent
Los objetivos descritos a continuación dependen del proceso y la planta a elegir donde se realizara el estudio, ya que la profundidad del estudio y limitaciones de tiempo  no permiten abarcarlos todos.

\subsection{Objetivo General}
\begin{itemize}
 \item Determinar y prevenir los posibles riesgos de un proceso en una empresa.
\end{itemize}

\subsection{Objetivos Específicos}
\begin{itemize}
 \item Analizar el proceso productivo a seleccionar de la empresa.
 \item Detección de riesgos asociados al proceso por medio de la matriz de riesgo \textbf{GTC 45 2010}.
 \item Calcular los riesgos formales abarcados en el curso a partir de mediciones en el proceso seleccionado.
 \item Analizar todas las \textbf{NTC} y \textbf{NTP} aplicables a los riesgos detectados.
 \item Diseño de un plan de gestión de los riesgos detectados.
\end{itemize}

\section{Metodología}
\noindent
Para el buen desarrollo del proyecto, se llevarán a cabo mediciones exactas y precisas según el proceso productivo, utilizando los instrumentos  y técnicas recomendados durante la cátedra. Con ayuda de estas mediciones y tomando como soporte técnico la norma adecuada según el caso a estudiar, se dará paso a un análisis para la identificación de riesgos potenciales, sus causas, sus posibles consecuencias, y se efectuarán recomendaciones para mitigarlos, corregirlos o prevenirlos. De tal forma que se puedan generar protocolos de operación, para preservar la salud del operario al momento de realizar su labor.

\section{Alcance}
\noindent
El alcance del estudio de la seguridad industrial de una empresa se limita a la entrega de un informe a la empresa y al docente de la materia. Dicho informe incluye el desarrollo de todos los objetivos, además de algunas recomendaciones para la mejoría de la seguridad al igual que para futuras investigaciones.\\
A demás el buen desarrollo de este contenido permitirá realizar grandes análisis de riesgo en cualquier empresa.

\section{Cronograma}
\noindent
El cronograma de visitas está dividido en semanas, las visitas se realizaran únicamente días hábiles  a acordar con la empresa, cuando la empresa está en funcionamiento, con un estimado de 4 horas por semana, el cronograma   está sujeto a cambio por posibles problemas con las disponibilidades de la empresa o estudiantes:
\begin{enumerate}
  \item Semana del 4 al 10 de noviembre de 2013
  \begin{itemize}
   \item Visita general de todas las instalaciones de la empresa. \item Acordar Cronograma de visitas.
   \item Ética profesional frente a la seguridad industrial.
   \item Teoría general del riesgo industrial.
   \item Identificación y análisis de riesgos industriales.
   \item Cálculos formales en seguridad industrial.
   \item Gestión de riesgos.
  \end{itemize}
  \item 11 a 17 de noviembre de 2013.
  \begin{itemize} 
   \item Cálculos formales en seguridad industrial.
   \item Gestión de riesgos.
  \end{itemize}
  \item 18 a 24 de noviembre de 2013.
  \begin{itemize}
   \item Cálculos formales en seguridad industrial.
   \item Gestión de riesgos.
   \item Riesgo físico.
  \end{itemize}
  \item 25 de noviembre a 1 de diciembre de 2013.
  \begin{itemize}
   \item Riesgo físico
  \end{itemize}
  \item 2 a 8 de diciembre de 2013.
  \begin{itemize}
   \item Riesgo físico.
  \end{itemize}
  \item 9 a 15 de diciembre de 2013.
  \begin{itemize}
   \item Riesgo físico
  \end{itemize}
  \item 16 a 24 de diciembre de 2013.
  \begin{itemize}
   \item Riesgo ergonómico.
  \end{itemize}
\end{enumerate}  
  
\section{Recursos disponibles}
Para la realización del análisis de riesgo, se tendrá en cuenta: \begin{itemize} 
  \item Textos científicos o técnicos acerca del sector o proceso a estudiar.
  \item Normas NTC y NTP.
  \item Cátedras y contenido del curso.
\end{itemize}
\noindent
En la medida en que avance el análisis de riesgo y la asignatura, se incluirán nuevos elementos a tener en cuenta.

\bibliographystyle{ieeetran}
\begin{thebibliography}{99}

\bibitem{page1} \url{https://sites.google.com/site/siun20123/home}, Pagina del curso Seguridad Industrial $2013-3$, visitada el 12 de Agosto de 2013.

\end{thebibliography}
\end{document}